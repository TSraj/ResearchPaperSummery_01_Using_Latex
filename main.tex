\documentclass{article}
\usepackage[utf8]{inputenc}

\title{Research Paper Summery-01}
\author{Md.Tanvir sadik Raj}
\date{December 10, 2020}

\begin{document}

\maketitle

\section{Introduction}
Paper:3.1/01 \newline \newline
Year:June 2007 \newline \newline
Title:Global Software Engineering: The Future of Socio-technical Coordination \newline \newline
Location: United States \newline

\section{Abstract}
Globally distributed project are becoming the norm for almost all the software companies all over the world. It’s giving many facility but also impairing many critical coordination mechanism. This paper emphasizes on this matter how this problem should be resolved for future global development and the challenges that are standing in the way of this vision. Mainly there are four critical challenges in this area :- \newline \newline
1.	Software architecture \newline
2.	Eliciting and communicating requirements \newline
3.	Environment and tools \newline
4.	Orchestrating global development \newline \newline
a systematic and well organized mechanism is required for the purpose of well globally development of any specific software projects. \newline
\section{Concept of Global Software  Development(GSD)}
It’s no longer unusual for a software project to have multiple teams in different continent. For the sake of getting an excellent work different highly skilled people from different location join on a specific team to complete the work. This method of working is effective but it arrives many problems. This paper is focused on technical coordination in geographical distributed projects. The paper also focuses on those features of software projects that influence the need to coordinate. The paper does not business related question or legal agreement decision. \newline \newline
In a traditional co-located project teams already have experienced what is the coordination process among them for an effective  work. Persons with different expertise work on specific portion and they can communicate over a formal meeting. But it’s not possible for global development process.  The fundamental problem in a GSD is that function to coordinate the work in a co-located setting are absent. Geographical distance make a huge impact even for a small decision & this also affect the development speed. Much less communication, less effective communication is the main motive of GSD. \\
\subsection*{•	Lack of awareness : For the reason for being at different sites they can share little information. This little context makes a huge different of works. This leads to misunderstanding of communication. } \\
\subsection*{•	Incompatibilities: Sites often differ in development tools, process, practices, informal work habits, corporate culture. These difference may often be incompatible and it’s may lead to misunderstanding. Sometimes this may arise some problems. }
\section{Key Points}
\subsection*{	Software Architecture: software architecture is not only the software design but also a important means of coordinating a software projects. Adapting to a good architecture tends to guide developers to there expected working desire. }
\subsection*{	Eliciting And Communicating Requirements: Getting the requirements right and dealing with unstable requirements are notoriously difficult. In the global development context, the inherent difficulty of achieving a shared understanding of the requirements is amplified, both because of loss of context and loss of communication bandwidth.}
\subsection*{	Environment And Tools: software engineering has long focused on developing & deploying tools to assist with coordinating software projects like version control and change management. GSD research on environment and tools focuses on extending these standard tools or integrating with these.}
\subsection*{	Orchestrating global development: The methods, practices and structure often used for general projects may not adequate for GSD. Typical ways relay heavily on frequent communication shared knowledge & common history.}
\section{Conclusion and research challenges for GSD  }
In order to make substantial progress in GSD deep understanding of coordination is required for any proper solution and also the factors that are required for any specific projects. The prediction capability is required in order to fix the methods tools and architecture which are needed in order to complete any software projects. We have a pressing need for a good theories that will provide a sound basis for reasoning about tradeoffs and predicting outcomes. \\
What practice are effective when? We do not know much about when various point on GSD practice are effective nor not. The truth is really somewhere in between. Nor is it the case that the decision is to adapt a practice can be considered in total isolation  from all other practice.  
\end{document}
